\documentclass{scrartcl}
\usepackage[utf8]{inputenc}
\usepackage{ngerman}
\usepackage{enumerate}
\usepackage{lmodern}
\usepackage{amsmath}
\usepackage{hyperref}
 
\title{MineCrawler}
\subtitle{Query on XML file}
\author{
  Tabea Treutwein\\
  \texttt{tabea.treutwein@st.ovgu.de}\\
  \and
  Dennis Meyer\\
  \texttt{dennis.meyer@st.ovgu.de}\\
  \and
  Alexander Simmer\\
  \texttt{asimmer@st.ovgu.de}\\
  \and
  Kilian Gärtner\\
  \texttt{kilian.gaertner@st.ovgu.de}\\
}

\date{\today}
 
\begin{document}
\maketitle
 
\section{Task}
The task was to do a search query on a document collection. The collection was in XML encoded and was given. The user should be able to enter a search query, which should be executed on the parsed collection. 
How to parse the documents was not given, so we choose the standard ava implementation JAXB to read and write XML. \\
For the indexing and executing the parsing was the library Apache Lucene given.

%\section{Aufgabe}
%Die Aufgabe bestand daraus, eine Suchanfrage auf eine Dokumentensammlung anzuwenden. Die Dokumentensammlung lag als XML kodierte. Diese sollte dann geparsed werden und auf den erhaltenen Dokumenten die Suchanfrage ausgeführt werden. Das Parsing war offen gelassen, wir haben uns für die Java Standardimplementierung JAXB entschieden. \\
%Für die Suchanfrage sollte mit der Bibliothek "Apache Lucene" ein Index erstellt und dann auf diesem die Suchanfrage ausgeführt werden.

\section{Usage}
The user can start the java programm without any arguments, he will be asked for the options like the number of results or the query itself. This is so simple, we will not explain this here, just try it :) \\
If the user doesn't want to use this simple interface, he can control the program with arguments. The order of the arguments is not relevant, the last parameter will be interpreted as the query.
The followning options are possible

\begin{description}
  \item[-l NUMBER] \hfill \\
  The maximum size of the result list. Standard value is 10.
  \item[-c] \hfill \\
  Should the result list printed on the console instead write in a file(will be created by default)
  \item[-f PATH] \hfill \\
  The path to the xml file to parse. Standard is the given xml file (in the jar) reut2-000.xml
\end{description}

%Wird das Java Programm ohne Argumente von der Konsole gestartet, wird der User nach den verschiedenen Parametern wie maximale Anzahl der Resultate oder der Query gefragt. Dies ist intuitiv und wird hier nicht weiter erklärt.\\
%Das Java Programm kann auch rein von der Konsole über Argumente ausgeführt werden. 
%Die Reihenfolge der Optionen ist egal, das letzte Argument wird als Query gewertet.
%Hier eine Auflistung der möglichen Optionen:\\\\
%-l [ZAHL] Maximale Anzahl der Resultate. Standardmäßig werden die Top 10 Resultate ausgegeben.\\
%-c Ausgabe der Ergebnisse auf der Konsole statt in einer Datei. Standardmäßig erfolgt die Ausgabe in eine Datei resultX.xml\\
%-f [PFAD] Pfad zur XML Datei. Standard ist die mitgelieferte reut2-000.xml Datei in der .jar\\
%Die Ausgabe erfolgt immer in eine XML Datei und das Format dieser entspricht der mitgelieferten reut2-000.xml.

\subsection{Query format}
The query format is defined by the used Library Apache Lucene and can be found \href{http://lucene.apache.org/core/4_6_0/queryparser/org/apache/lucene/queryparser/classic/package-summary.html#package_description}{on this website}.  \\
The possible search fields are 
\begin{itemize}
	\item id
	\item date
	\item title
	\item body
\end{itemize}

%\section{Bedienung}
%Wird das Java Programm ohne Argumente von der Konsole gestartet, wird der User nach den verschiedenen Parametern wie maximale Anzahl der Resultate oder der Query gefragt. Dies ist intuitiv und wird hier nicht weiter erklärt.\\
%Das Java Programm kann auch rein von der Konsole über Argumente ausgeführt werden. 
%Die Reihenfolge der Optionen ist egal, das letzte Argument wird als Query gewertet.
%Hier eine Auflistung der möglichen Optionen:\\\\
%-l [ZAHL] Maximale Anzahl der Resultate. Standardmäßig werden die Top 10 Resultate ausgegeben.\\
%-c Ausgabe der Ergebnisse auf der Konsole statt in einer Datei. Standardmäßig erfolgt die Ausgabe in eine Datei resultX.xml\\
%-f [PFAD] Pfad zur XML Datei. Standard ist die mitgelieferte reut2-000.xml Datei in der .jar\\
%Die Ausgabe erfolgt immer in eine XML Datei und das Format dieser entspricht der mitgelieferten reut2-000.xml.

\section{Libraries}
The used libraries for this project are

\begin{enumerate}
	\item \href{http://lucene.apache.org/}{Apache Lucene} - Creating an index and executing the query
	\item \href{http://www.joda.org/joda-time/}{JodaTime} - For easier date handeling(normal java implementation isn't the best...)
	\item \href{http://commons.apache.org/proper/commons-cli/}{Apache Commons CLI} - Easy parsing of console arguments in POSIX style
\end{enumerate}

%\section{Bibliotheken}
%Die benutzen Bibliotheken sind:
%\begin{enumerate}
%	\item \href{http://lucene.apache.org/}{Apache Lucene} - Erstellung des Index und Ausführung der Suchanfrage
%	\item \href{http://www.joda.org/joda-time/}{JodaTime} - Für Verarbeitung von Datumsangaben im Dokument
%	\item \href{http://commons.apache.org/proper/commons-cli/}{Apache Commons CLI} - Auswerten der Argumente von der Konsole
%\end{enumerate}

\end{document}