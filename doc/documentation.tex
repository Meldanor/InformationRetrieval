\documentclass{scrartcl}
\usepackage[utf8]{inputenc}
\usepackage{ngerman}
\usepackage{enumerate}
\usepackage{lmodern}
\usepackage{amsmath}
\usepackage{hyperref}
 
\title{MineCrawler}
\subtitle{Query on XML file}
\author{
  Tabea Treutwein\\
  \texttt{tabea.treutwein@st.ovgu.de}\\
  \and
  Dennis Meyer\\
  \texttt{dennis.meyer@st.ovgu.de}\\
  \and
  Alexander Simmer\\
  \texttt{asimmer@st.ovgu.de}\\
  \and
  Kilian Gärtner\\
  \texttt{kilian.gaertner@st.ovgu.de}\\
}

\date{\today}
 
\begin{document}
\maketitle
 
\section{Task}
The main task of this project was to perform a search query on a collection of documents, which is given as a XML document. The user should be able to perform search queries on this collection.

Since no parsing method was specified, we chose the standart Java Library JAXB to read and write XML files. For the indexing and parsing of the collection we were required to use the Apache Lucene library.

\section{Usage}
The Java executable accepts the following arguments:

\begin{description}
  \item[-l NUMBER] \hfill \\
  The maximum size of the result list. Standard value is 10.
  \item[-c] \hfill \\
  Should the result list printed on the console instead write in a file(will be created by default)
  \item[-f PATH] \hfill \\
  The path to the xml file to parse. Standard is the given xml file (in the jar) reut2-000.xml
\end{description}

Note that the order of the arguments does not matter. In case none of these arguments are specified, the user will be prompted to enter additional parameters (i.e. number of results, search query). Those are quite self-explanatory and will therefore not be elaborated here.

\subsection{Query format}
The query format is defined by the used Library Apache Lucene and can be found \href{http://lucene.apache.org/core/4_6_0/queryparser/org/apache/lucene/queryparser/classic/package-summary.html#package_description}{on this website}.  \\
The possible search fields are 
\begin{itemize}
	\item id
	\item date
	\item title
	\item body
\end{itemize}

%Eventuell anders formatieren, Itemization ist hier eher unpassend, es sei denn ihr wollt noch Beschreibungen wie unter "Usage" hinzufügen.

\section{Libraries}
The used libraries for this project are

\begin{enumerate}
	\item \href{http://lucene.apache.org/}{Apache Lucene} - Creating an index and executing the query
	\item \href{http://www.joda.org/joda-time/}{JodaTime} - For easier date handeling(normal java implementation isn't the best...)
	\item \href{http://commons.apache.org/proper/commons-cli/}{Apache Commons CLI} - Easy parsing of console arguments in POSIX style
\end{enumerate}

%\section{Bibliotheken}
%Die benutzen Bibliotheken sind:
%\begin{enumerate}
%	\item \href{http://lucene.apache.org/}{Apache Lucene} - Erstellung des Index und Ausführung der Suchanfrage
%	\item \href{http://www.joda.org/joda-time/}{JodaTime} - Für Verarbeitung von Datumsangaben im Dokument
%	\item \href{http://commons.apache.org/proper/commons-cli/}{Apache Commons CLI} - Auswerten der Argumente von der Konsole
%\end{enumerate}

\end{document}